\documentclass[12pt]{article}
\usepackage{synttree}
\begin{document}

\section{}

\subsection*{(a)}

Shown below are two different parse trees for the same sentence.

~

\synttree
[S
	[NP
		[CN
			[A
				[a]
			]
			[NN
				[cat]
			]
		]
	]
	[VP
		[CV
			[V
				[scratches]
			]
			[NP
				[CN
					[A
						[the]
					]
					[NN
						[baby]
					]
				]
				[PP
					[PR
						[with]
					]
					[CN
						[A
							[a]
						]
						[NN
							[nail]
						]
					]
				]
			]
		]
	]
]

~

\synttree
[S
	[NP
		[CN
			[A
				[a]
			]
			[NN
				[cat]
			]
		]
	]
	[VP
		[CV
			[V
				[scratches]
			]
			[NP
				[CN
					[A
						[the]
					]
					[NN
						[baby]
					]
				]
			]
		]
		[PP
			[PR
				[with]
			]
			[CN
				[A
					[a]
				]
				[NN
					[nail]
				]
			]
		]
	]
]

\subsection*{(b)}

The part of the sentence 'with a nail' could refer to either the cat or the baby. If it refers to the cat, then the sentence means that the cat is using the nail to scratch the baby. If it refers to the baby, then the sentence means that the baby has a nail in their possession, and the cat is also scratching him. 

\section{}

For the grammars that I have written, I have used 'ID' as a nonterminal symbol, which can be used to produce a list of identifiers that suit this question (a, b, c, and d). 

\subsection*{(a)}

Let $ G = \{T, N, P, S\} $, where $ T = \{a, b, c, d, /, * \} $, $ N = \{S, E, T, ID \} $ and productions given by:

$$ S \rightarrow E $$

$$ E \rightarrow E~/~T~\mid~T $$

$$ T \rightarrow T~*~T~\mid~ID $$

$$ ID \rightarrow a~\mid~b~\mid~c~\mid~d $$

The parse tree made according to this grammar is shown below.

~

\synttree
[S
	[E
		[E
			[T
				[T
					[ID
						[a]
					]
				]
				[*]
				[T
					[ID
						[b]
					]
				]
			]
		]
		[/]
		[T
			[T
				[ID
					[c]
				]
			]
			[*]
			[T
				[ID
					[d]
				]
			]
		]
	]
]

\subsection*{(b)}

Let $ G = \{T, N, P, S\} $, where $ T = \{a, b, c, d, /, * \} $, $ N = \{S, E, T, ID \} $ and productions given by:

$$ S \rightarrow E $$

$$ E \rightarrow E~*~T~\mid~T $$

$$ T \rightarrow T~/~T~\mid~ID $$

$$ ID \rightarrow a~\mid~b~\mid~c~\mid~d $$

The parse tree made according to this grammar is shown below.

\synttree
[S
	[E
		[E
			[E
				[T
					[ID
						[a]
					]
				]
			]
			[*]
			[T
				[T
					[ID
						[b]
					]
				]
				[/]
				[T
					[ID
						[c]
					]
				]
			]
		]
		[*]
		[T
			[ID
				[d]
			]
		]
	]
]

\subsection*{(c)}

Let $ G = \{T, N, P, S\} $, where $ T = \{a, b, c, d, /, * \} $, $ N = \{S, E, T, ID \} $ and productions given by:

$$ S \rightarrow E $$

$$ E \rightarrow E~*~T~\mid~E~/~T~\mid~T $$

$$ T \rightarrow ID $$

$$ ID \rightarrow a~\mid~b~\mid~c~\mid~d $$

The parse tree made according to this grammar is shown below.

\synttree
[S
	[E
		[E
			[E
				[E
					[T
						[ID
							[a]
						]
					]
				]
				[*]
				[T
					[ID
						[b]
					]
				]
			]
			[/]
			[T
				[ID
					[c]
				]
			]
		]
		[*]
		[T
			[ID
				[d]
			]
		]
	]
]

\subsection*{(d)}

Let $ G = \{T, N, P, S\} $, where $ T = \{a, b, c, d, /, * \} $, $ N = \{S, E, T, ID \} $ and productions given by:

$$ S \rightarrow E $$

$$ E \rightarrow T~*~E~\mid~T~/~E~\mid~T $$

$$ T \rightarrow ID $$

$$ ID \rightarrow a~\mid~b~\mid~c~\mid~d $$

The parse tree made according to this grammar is shown below.

\synttree
[S
	[E
		[T
			[ID
				[a]
			]
		]
		[*]
		[E
			[T
				[ID
					[b]
				]
			]
			[/]
			[E
				[T
					[ID
						[c]
					]
				]
				[*]
				[E
					[T
						[ID
							[d]
						]
					]
				]
			]
		]
	]
]

\section{}

I start by proving the inclusion $ L(G) \subseteq  \{a^nbc^n \mid n \geq 0\}$ by considering all productions and showing, by induction, that all sentences derived are of the form $ a^nbc^n $. The first production, $ S \rightarrow A $, does not generate any new nonterminal symbols and so does not actually advance the derivation. So, it will not be considered. For the remaining productions, I will assume that the A on the right hand side is already of the form $ a^nbc^n $ and show that the resulting left hand side is also of the correct form. For the second production, $ A \rightarrow b $, which serves as the base case, we have $ A \rightarrow b = a^0bc^0 $, so the inclusion holds here. For the last production, $ A \rightarrow  aAc$, assume that the A on the right hand side of the production is already of the form $ a^nbc^n $. This is the induction hypothesis. Then, $ aAc = a^{n+1}bc^{n+1} $, which is also of the correct form. So, the inclusion holds for all productions. Thus, $ L(G) \subseteq  \{a^nbc^n \mid n \geq 0\}$.

Next, I prove the inclusion $ \{a^nbc^n \mid n \geq 0\} \subseteq L(G) $ by showing that every word of the form $ a^nbc^n $ can be generated by G. I will show this using induction on n. Of course, the base case of $ a^0bc^0 = b $ can be derived as $ S \rightarrow A \rightarrow b $. Next, assume that $ a^nbc^n $ can be derived using the productions in G. Then, we have the derivation $ S \rightarrow^* a^nbc^n $. Because of the structure of the grammar, anything generated from S must be an A. So, I apply the production $ A \rightarrow aAc $. The derivation $ S \rightarrow^* a^nbc^n \rightarrow aa^nbc^nc = a^{n+1}bc^{n+1} $ shows how $a^{n+1}bc^{n+1}$ is generated (which confirms the induction hypothesis) and so the inclusion holds for all values of n. Thus, $ \{a^nbc^n \mid n \geq 0\} \subseteq L(G) $. 

Because I have shown mutual inclusion, I can conclude that $ L(G) = \{a^nbc^n \mid n \geq 0\} $. 


\end{document}